\documentclass[11pt,spanish]{article}

\usepackage{listings}             
\usepackage{anysize} 
\usepackage{graphicx}
\usepackage[spanish]{babel}
\usepackage[utf8]{inputenc}
\usepackage{xcolor}
\usepackage{wrapfig}
\renewcommand\lstlistingname{Código}


\lstset{language=Python}
\marginsize{1cm}{1cm}{2cm}{2cm}
\selectlanguage{spanish}
\lstset{
language=Python,
 backgroundcolor=\color{red!75!green!50!blue!25},
 frame=single,
literate=
  {á}{{\'a}}1 {é}{{\'e}}1 {í}{{\'i}}1 {ó}{{\'o}}1 {ú}{{\'u}}1
  {Á}{{\'A}}1 {É}{{\'E}}1 {Í}{{\'I}}1 {Ó}{{\'O}}1 {Ú}{{\'U}}1
  {à}{{\`a}}1 {è}{{\`e}}1 {ì}{{\`i}}1 {ò}{{\`o}}1 {ù}{{\`u}}1
  {À}{{\`A}}1 {È}{{\'E}}1 {Ì}{{\`I}}1 {Ò}{{\`O}}1 {Ù}{{\`U}}1
  {ä}{{\"a}}1 {ë}{{\"e}}1 {ï}{{\"i}}1 {ö}{{\"o}}1 {ü}{{\"u}}1
  {Ä}{{\"A}}1 {Ë}{{\"E}}1 {Ï}{{\"I}}1 {Ö}{{\"O}}1 {Ü}{{\"U}}1
  {â}{{\^a}}1 {ê}{{\^e}}1 {î}{{\^i}}1 {ô}{{\^o}}1 {û}{{\^u}}1
  {Â}{{\^A}}1 {Ê}{{\^E}}1 {Î}{{\^I}}1 {Ô}{{\^O}}1 {Û}{{\^U}}1
  {œ}{{\oe}}1 {Œ}{{\OE}}1 {æ}{{\ae}}1 {Æ}{{\AE}}1 {ß}{{\ss}}1
  {ű}{{\H{u}}}1 {Ű}{{\H{U}}}1 {ő}{{\H{o}}}1 {Ő}{{\H{O}}}1
  {ç}{{\c c}}1 {Ç}{{\c C}}1 {ø}{{\o}}1 {å}{{\r a}}1 {Å}{{\r A}}1
  {€}{{\EUR}}1 {£}{{\pounds}}1
}


\title{\vspace{-3cm}\begin{flushleft}\textbf{Actividad 2}\end{flushleft}}
\author{\hspace{-9.6cm}\textsc{Andrés Ignacio Rodríguez Mendoza}}
\date{}

\begin{document}

\begin{wrapfigure}{r}{0.2\textwidth}
  \begin{center}
   \vspace{-5.4cm} \includegraphics[width=0.15\textwidth]{uni}
  \end{center}
\end{wrapfigure}

\maketitle
\begin{center}
\rule{\textwidth}{1pt}
\end{center}

La sencilla sintaxis del lenguaje de programación Python, favorece la lectura del código. Este lenguaje soporta orientación a objetos, programación imperativa y programación funcional.
Posee licencia de código abierto, llamada Python Software Foundation License.\\ \\
La presente práctica requiere modificar y correr diversos códigos en Python para comprender su dinámica y aprender algunos comandos.\\


\begin{lstlisting}
import this
\end{lstlisting} 

\begin{tabular}{|l}
\begin{minipage}{3in}

\begin{verbatim}
The Zen of Python, by Tim Peters

Beautiful is better than ugly.
Explicit is better than implicit.
Simple is better than complex.
Complex is better than complicated.
Flat is better than nested.
Sparse is better than dense.
Readability counts.
Special cases aren't special enough to break the rules.
Although practicality beats purity.
Errors should never pass silently.
Unless explicitly silenced.
In the face of ambiguity, refuse the temptation to guess.
There should be one-- and preferably only one --obvious way to do it.
Although that way may not be obvious at first unless you're Dutch.
Now is better than never.
Although never is often better than *right* now.
If the implementation is hard to explain, it's a bad idea.
If the implementation is easy to explain, it may be a good idea.
Namespaces are one honking great idea -- let's do more of those!
\end{verbatim}
\end{minipage}
\end{tabular} \\

\section*{Códigos}
\begin{lstlisting}[caption=caida.py] 

import math 

h = float(input("Proporciona la altura de la torre: "))

t = sqrt(2*h/9.81)


print("EL tiempo de caída es", t, "segundos")

\end{lstlisting}



\begin{lstlisting}[caption=altura.py] 

import math

G=6.67e-11
M=5.97e24
R=6371000
pi=3.1416

t= float(input("Periodo del satélite:"))
T= t*60
h=( (G*M*T*T) / (4*pi*pi) )**(1/3) - R

print ("Altura del satélite:", h, "metros.")

\end{lstlisting}


\begin{lstlisting}[caption=altura.py] 

import math

G=6.67e-11
M=5.97e24
R=6371000
pi=3.1416

t= float(input("Periodo del satélite:"))
T= t*60
h=( (G*M*T*T) / (4*pi*pi) )**(1/3) - R

print ("Altura del satélite:", h, "metros.")

\end{lstlisting}



\begin{lstlisting}[caption=Polar.py] 

from math import sin,cos,pi

r = float(input("Introduce r: "))

d = float(input("Ingresa theta en grados: "))

theta = d*pi/180

x = r*cos(theta)

y = r*sin(theta)

print("x =",x," y =",y)

\end{lstlisting}

\begin{lstlisting}[caption=esfericas.py] 

from math import sin,cos,pi

r = float(input("Introduce r: "))

d = float(input("Ingresa theta en grados: "))

f = float(input("Ingresa phi en grados: "))

theta = d*pi/180
phi = f*pi/180 


x = r*sin(theta)*cos(phi)

y = r*sin(theta)*sin(phi)

z = r*cos(theta)

print("x =",x," y =",y, " z = ",z)

\end{lstlisting}


\begin{lstlisting}[caption=EvenOdd.py] 

print("Enter two integers, one even, one odd.")
m = int(input("Enter the first integer: "))
n = int(input("Enter the second integer: "))
while (m+n)%2==0:
    print("One must be even and the other odd.")
    m = int(input("Enter the first integer: "))
    n = int(input("Enter the second integer: "))
print("The numbers you chose are",m,"and",n)

\end{lstlisting}

\begin{lstlisting}[caption=Fibonacci.py] 

f1,f2 = 1,1

while f2<1000:

      print(f2)

      f1,f2 = f2,f1+f2
      
\end{lstlisting}

\begin{lstlisting}[caption=Catalan.py] 

n=0
C=1

while C<=1000000:

      print(C)

      C= C*2*(2*n+1)/(n+2)
      n=n+1     
      
\end{lstlisting}

\end{document}