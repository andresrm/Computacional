\documentclass[11pt,spanish]{article}

\usepackage{listings}             
\usepackage{anysize} 
\usepackage{graphicx}
\usepackage[spanish]{babel}
\usepackage[utf8]{inputenc}
\usepackage{xcolor}
\usepackage{wrapfig}


\renewcommand\lstlistingname{Código}
\lstset{language=Python}
\marginsize{1cm}{1cm}{2cm}{2cm}
\selectlanguage{spanish}
\lstset{
language=Python,
 backgroundcolor=\color{red!75!green!50!blue!25},
 frame=single,
literate=
  {á}{{\'a}}1 {é}{{\'e}}1 {í}{{\'i}}1 {ó}{{\'o}}1 {ú}{{\'u}}1
  {Á}{{\'A}}1 {É}{{\'E}}1 {Í}{{\'I}}1 {Ó}{{\'O}}1 {Ú}{{\'U}}1
  {à}{{\`a}}1 {è}{{\`e}}1 {ì}{{\`i}}1 {ò}{{\`o}}1 {ù}{{\`u}}1
  {À}{{\`A}}1 {È}{{\'E}}1 {Ì}{{\`I}}1 {Ò}{{\`O}}1 {Ù}{{\`U}}1
  {ä}{{\"a}}1 {ë}{{\"e}}1 {ï}{{\"i}}1 {ö}{{\"o}}1 {ü}{{\"u}}1
  {Ä}{{\"A}}1 {Ë}{{\"E}}1 {Ï}{{\"I}}1 {Ö}{{\"O}}1 {Ü}{{\"U}}1
  {â}{{\^a}}1 {ê}{{\^e}}1 {î}{{\^i}}1 {ô}{{\^o}}1 {û}{{\^u}}1
  {Â}{{\^A}}1 {Ê}{{\^E}}1 {Î}{{\^I}}1 {Ô}{{\^O}}1 {Û}{{\^U}}1
  {œ}{{\oe}}1 {Œ}{{\OE}}1 {æ}{{\ae}}1 {Æ}{{\AE}}1 {ß}{{\ss}}1
  {ű}{{\H{u}}}1 {Ű}{{\H{U}}}1 {ő}{{\H{o}}}1 {Ő}{{\H{O}}}1
  {ç}{{\c c}}1 {Ç}{{\c C}}1 {ø}{{\o}}1 {å}{{\r a}}1 {Å}{{\r A}}1
  {€}{{\EUR}}1 {£}{{\pounds}}1
}


\title{\vspace{-3cm}\begin{flushleft}\textbf{Actividad 10}\end{flushleft}}
\author{\hspace{-9.6cm}\textsc{Andrés Ignacio Rodríguez Mendoza}}
\date{}

\begin{document}

\begin{wrapfigure}{r}{0.2\textwidth}
  \begin{center}
   \vspace{-5.4cm} \includegraphics[width=0.15\textwidth]{uni}
  \end{center}
\end{wrapfigure}

\maketitle  
\begin{center}
\rule{\textwidth}{1pt}
\end{center}

$$\alpha$$

\section*{Código}

\begin{lstlisting}

# Double pendulum formula translated from the C code at
# http://www.physics.usyd.edu.au/~wheat/dpend_html/solve_dpend.c

from numpy import sin, cos
import numpy as np
import matplotlib.pyplot as plt
import scipy.integrate as integrate
import matplotlib.animation as animation

G = 9.8  # acceleration due to gravity, in m/s^2
L1 = 1.0  # length of pendulum 1 in m
L2 = 1  # length of pendulum 2 in m
M1 = 1.0  # mass of pendulum 1 in kg
M2 = 0.0  # mass of pendulum 2 in kg

 
    
def derivs(state, t):

    dydx = np.zeros_like(state)
    dydx[0] = state[1]

    del_ = state[2] - state[0]
    den1 = (M1 + M2)*L1 - M2*L1*cos(del_)*cos(del_)
    dydx[1] = (M2*L1*state[1]*state[1]*sin(del_)*cos(del_) +
               M2*G*sin(state[2])*cos(del_) +
               M2*L2*state[3]*state[3]*sin(del_) -
               (M1 + M2)*G*sin(state[0]))/den1

    dydx[2] = state[3]

    den2 = (L2/L1)*den1
    dydx[3] = (-M2*L2*state[3]*state[3]*sin(del_)*cos(del_) +
               (M1 + M2)*G*sin(state[0])*cos(del_) -
               (M1 + M2)*L1*state[1]*state[1]*sin(del_) -
               (M1 + M2)*G*sin(state[2]))/den2

    return dydx

# create a time array from 0..100 sampled at 0.05 second steps
dt = 0.05
t = np.arange(0.0, 10, dt)

# th1 and th2 are the initial angles (degrees)
# w10 and w20 are the initial angular velocities (degrees per second)
th1 = 135.0
w1 = 0.0
th2 = -10.0
w2 = 0.0

# initial state
state = np.radians([th1, w1, th2, w2])

# integrate your ODE using scipy.integrate.
y = integrate.odeint(derivs, state, t)

x1 = L1*sin(y[:, 0])
y1 = -L1*cos(y[:, 0])

x2 = L2*sin(y[:, 2]) + x1
y2 = -L2*cos(y[:, 2]) + y1

#Y=integrate.odeint(pend,[th1,w1],t)

th= y[:, 0]
w = y[:,1]

# gráficas
fig1 = plt.figure(figsize=(6, 6.1))

ax1 = fig1.add_subplot(212, autoscale_on=False, xlim=(-2, 2), ylim=(-1.05, 1.05))
ax2 = fig1.add_subplot(211, autoscale_on=False, xlim=(-2, 2), ylim=(-1.05, 1.05))

ax1.axis('off')
ax2.axis('off')

ax2.plot([-2,2],[0,0],'k',lw=1)
plt.title(r'$\theta _ 0 =  %i  ^o \qquad \omega _ 0 = %s$' %(th1,w1), fontsize=18)

line1,   = ax1.plot([], [], '*', color = 'y',    ms=13)
line1_1, = ax1.plot([], [], '-', color='g',      lw=0.8)
line1_0, = ax1.plot([], [], 'o', color='k',      lw=2)
line2,   = ax2.plot([], [], 'H', color='sienna', ms=8)
line2_1, = ax2.plot([], [], '-', color='c',      lw=0.4)
line2_0, = ax2.plot([], [], 'o', color='k',      lw=2)



time_template = 'time = %.1fs'
time_text = ax2.text(0.05, 0.9, '', transform=ax2.transAxes)


def init():
    line1.set_data([], [])
    time_text.set_text('')
    return line1, time_text

def init2():
    line2.set_data([], [])
    time_text.set_text('')
    return line1, time_text


def animate(i):
    thisx = [0, x1[i]]
    thisy = [0, y1[i]]
    
    thisth = [0, th[i]/np.pi]
    thisw = [0, w[i]/ (6*np.sqrt(2*G*(1-np.cos(th1))))]

    line1.set_data(thisx, thisy)
    line1_1.set_data(thisx, thisy)
    line1_0.set_data(0, 0)
    line2.set_data(thisth, thisw)
    line2_0.set_data(0, 0)
    line2_1.set_data(th/np.pi,w/ (6 * np.sqrt(2*G*(1-np.cos(th1)))))
#    time_text.set_text(time_template % (i*dt))
    return line1, line2, time_text


ani = animation.FuncAnimation(fig1,animate, np.arange(1, len(y)),
                              interval=25, blit=True, init_func=init)

ani.save('pendulum_%i_%s.mp4'%(th1,w1), fps=15)

plt.show()

\end{lstlisting}


\section*{Gáficas}

\centering






\end{document}
